%!TEX encoding = UTF-8 Unicode

\documentclass{article}

\usepackage{kotex}

\title{Management of Patients on Non–Vitamin K Antagonist Oral Anticoagulants in the Acute Care and Periprocedural Setting: A Scientific Statement From the American Heart Association}
\author{Circulation. 2017;135:e604-e633}
\date{}

\begin{document}

\maketitle

\section{Background}
\begin{itemize}
	\item  In clinical practice, there is still widespread uncertainty on how to manage patients on NOACs who bleed or who are at risk for bleeding.
	\item Clinical trial data related to NOAC reversal for bleeding and perioperative management are sparse, and recommendations are largely derived from expert opinion.
\end{itemize}

\section{Scope}
\begin{itemize}
	\item \textbf{Anticoagulated patients} are vulnerable to spontaneous, traumatic and perioperative bleeding.
	\item \textbf{VKA and NOAC}
	\item  This scientific statement offers practical suggestions for \textbf{providers who manage patients who are actively bleeding and who are at risk for bleeding in the acute care and periprocedural setting}.
\end{itemize}

\section{Process}
\begin{itemize}
	\item Members of this American Heart Association (AHA) \textbf{writing group} were selected for their diverse expertise in cardiovascular medicine, emergency medicine, critical care, neurology, surgery, and pharmacology.
	\item A \textbf{systematic search} of the literature for each subtopic was performed in PubMed and Ovid and was supplemented by review of bibliographies as well as manual searches of key articles.
	\item  Search term:\\
	dabigatran, apixaban, rivaroxaban, edoxaban, anticoagulation, reversal, antidote, atrial fibrillation, venous thromboembolism, bleeding, intracranial, cardioversion, catheterization, cardiac implantable devices, kidney injury, transition, switching, pharmacology, andexanet alfa, idarucizumab, ciraparantag, gastrointestinal, trauma, surgery, percutaneous coronary intervention, neuraxial anesthesia, stroke, and overdose.
	\item Members were instructed to cite contemporary guidelines and scientific statements where appropriate.
	\item The writing group did not assign formal classes of recommendation/level of evidence per the AHA Scientific Document Development Process recommendation that went into effect September 1, 2015. 
	\item Sections were then reviewed by another writing group member. 
	\item Section drafts were submitted to the writing group chair and co-chair and compiled into a single document. Web and teleconferences were convened to review and edit the full draft.
	\item The final document was submitted for independent peer review and approved for publication by the AHA Manuscript Oversight Committee on April 29, 2016.
\end{itemize}

\section{Topics}
\begin{itemize}
	\item Pharmacology of NOACS
	\item LABORATORY MEASUREMENT OF NOAC EFFECT
	\item NOAC REVERSAL
	\item MANAGEMENT OF LIFE-THREATENING BLEEDING
	\begin{itemize}
		\item ICH, GI bleeding
		\item Trauma
	\end{itemize}
	\item MANAGEMENT OF PATIENTS ON NOACS WHO ARE AT RISK FOR BLEEDING
	\begin{itemize}
		\item AKI, Stroke, Overdose
	\end{itemize}
	\item TRANSITIONING BETWEEN NOACS AND OTHER ANTICOAGULANTS IN THE ACUTE CARE SETTING
	\item PERIPROCEDURAL MANAGEMENT OF PATIENTS WHO TAKE NOACS
	\begin{itemize}
		\item Coronary intervention
		\item Cardioversion of AF
		\item Catheter ablation of AF
		\item Electronic device implantation
		\item CV surgery
		\item Non-CV surgery
		\item Neuraxial anesthesia
	\end{itemize}
\end{itemize}


	
\end{document}