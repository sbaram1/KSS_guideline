
\documentclass{paper}

\usepackage{kotex}
\usepackage{setspace}

\setstretch{1.2}


\begin{document}
\Large \textbf{개발계획 수립} \normalsize

\section{기존 진료지침 검색}
	\begin{itemize}
		\item Selection criteria
			\begin{itemize}
				\item From 2012 to 2016
				\item Written in English or Korean
				\item By Multi-disciplinary team
				\item Evidence-based method
			\end{itemize}
		\item 선택된 진료지침 검토
		\begin{itemize}
			\item Canadian guidelines (조용진)
			\item European guidelines (남효석)
			\item American guidelines (박광열)
		\end{itemize}
	\end{itemize}
	
\section{개발방법 결정}

\textbf{Hybrid}

%\textbf{Adaptation}
%\begin{itemize}
%	\item 기존 국내외 진료지침이 전체의 핵심질문을 모두 포함하는 경우
%	\item 관련 국내외 진료지침이 3-5년 이내에 개발되었고 결정적인 추가 근거가 없는 경우
%(근거 팽창 속도가 빠른 경우 3년)
%	\item 관련 국내외 진료지침(혹은 systematic review)이 근거기반 방법론*을 사용한 경우
%	\item 관련 국내외 진료지침이 국가(다국가 포함) 혹은 대표적인 학회에서 개발한 경우
%\end{itemize}

\end{document}