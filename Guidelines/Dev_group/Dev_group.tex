
\documentclass{paper}

\usepackage{kotex}
\usepackage{setspace}

\setstretch{1.2}

\begin{document}
\Large \textbf{개발그륩} \normalsize

\section{개발그룹 결성}
\subsection{관리위원회, Oversight committee}
최소 2개 이상의 임상과목 전문가와 방법론 전문가를 포함하도록 하며 위원장은 대한뇌졸중학회 이사장으로 한다. 위원구성은 다학제성, 대표성, 지속성의 원칙을 지키도록 한다.
주요업무의 범위는 다음과 같다:\\[3ex]
- 신규/개정 진료지침 주제, 범위, 대상집단 결정\\
- 검토계획의 수립 및 검토전문가 결정\\
- 진료지침 개발 각 단계에 맞는 위원 및 위원회 구성 승인 (세부주제 집필 책임자 포함) \\
- 실행위원회 결정사항의 최종 승인\\
- 진료지침 개발 후 관련학회 승인 범위 결정\\
- 집필위원회에 의해 완성된 초안의 내부검토\\
- 일반인을 위한 가이드라인 승인\\
- 최종안 확정

\subsection{실행위원회, Executive committee}
진료지침 개발 및 개작에 관한 다음과 같은 업무를 총괄한다.\\[3ex]
- 개발방법론 결정 및 매뉴얼 관리 - 집필위원회 구성 제안\\
- 개발계획의 수립\\
- 운영약관의 제정\\
- 지침의 배포 및 실행\\
- 진료지침 워크샵의 기획 및 개최

%\subsection{외부자문그룹}
%%이유경

\subsection{집필위원회, Writing committee}
구자성, 조용진, 남효석, 박광열, 이기정\\[3ex]

주제별 집필책임자를 중심으로 하는 팀을 구성하며 문헌검색과 자료 추출을 전담하는 정보검색전문가를 포함하는 것이 권장되며 다음과 같은 업무를 실행한다.\\[3ex]
- 기존 진료지침검토\\
- 근거문헌의 검색 및 근거표 작성\\
- 신규/개정 진료지침 집필\\
- 신규/개정 진료지침의 publication\\

추가 고려: \\
다학제성 고려 (FM, IM etc), \\
방법론 전문가(문헌검색, Systematic review)

%\subsection{근거평가그룹}
%
%\subsection{행정조직}

\vspace{30pt}
\section{이해관계 선언}
%USPSTF 이해상충에 대한 기준: 금액등

\subsection{이해 상충의 종류}
\begin{itemize}
	\item 재정적 이해  상충: 연구비, 자문료, 사례비, 주식, 지적재산권등
	\item 지적 이해 상충: 진료 헤게모니 등 
	\item 개인적인 이해 상충
%	\item 업무 활동에 있어 이해 상충(Conflicts of commitment)
\end{itemize}

\subsection{이해 상충에 대한 처리}
\begin{itemize}
	\item 공개
	\item 위원회에서 배제
	\item 해당 권고의 결정권 배제
\end{itemize}

\vspace{30pt}
\section{운영}
\subsection{합의 원칙의 결정}
토론\\
Delphi 

\subsection{저자 원칙 결정}
초안: \\
최종보고서: \\
저자 포함: 개발그룹 전체\\
저자 순서: 무순

\subsection{잠재적 승인기구 선정}
대한 뇌졸중학회\\
대한 신경과학회 ? (외부 검토시 포함)

\subsection{보급및 실행전략}
뇌졸중학회/신경과학회 홈페이지 게시\\
뇌졸중학회지, 신경과학회지 발표\\
소책자 발행하여 학회에서 배포

\end{document}