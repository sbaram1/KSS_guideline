
\documentclass{paper}

\usepackage{kotex}
\usepackage{setspace}
\usepackage{hyperref}
\usepackage{color}

\setstretch{1.2}


\begin{document}
\Large \textbf{범위결정과 핵심질문} \normalsize

\section{범위 결정}
\color{blue}
KHS의 예.
\begin{description}
	\item[Population] Adult with frequent and/or moderate-to-severe episodic migraine 
	\item[Intervention] Drug \\Anti-epileptic drug\\Beta blocker/CCB/ARB/ACE\\Anti-Depressant
	\item[Professional] Primary care physician, Nurse, and healthcare professionals
	\item[Outcome] The frequency of migraine prevention drug prescription as suggested
	\item[Healthcare setting] Primary care setting in Korea
\end{description}
\color{black}

\begin{description}
	\item[Population] Adult with ischemic stroke who requires surgery or invasive procedure		
	\item[Intervention] Withdrawal of antithrombotics
	\item[Professional] Primary care physician, Nurse, and healthcare professionals (뇌졸중을 진료하는 의사)
	\item[Outcome] 
	\item[Healthcare setting] Primary care setting in Korea
\end{description}

\newpage

\section{Key Questions}
\color{blue}
KHS의 예.
\begin{enumerate}
	\item Episodic migraine환자에서 예방치료를 고려해야 하는 요인들(두통빈도, 두통강도, 환자의 선호도, ADL에 대한 영향 등)은 무엇인가?
	\item Episodic migraine환자에서 예방약제를 선택할 때, 고려해야 하는 요인(임신, 수유, 및 우울증, 고혈압, 과체중 등 동반이환상태 등)은 무엇인가?
	\item Episodic migraine환자에서 예방약제를 어떻게 사용하는 것(용량조절방법, 두통일기의 사용, 효과평가시점, 효과평가방법 등)이 효과적일 것인가?
	\item 예방치료를 진행중인 episodic migraine환자에서 치료의 중단은 어떻게 결정해야 하는가?
	\item Episodic migraine환자에서 예방치료로 베타차단제(beta blocker; propranolol 등)를 사용하는 것이 타약제, 위약 또는 치료하지 않는 것에 비해 두통의 완화에 효과적인가?
	\item Episodic migraine환자에서 예방치료로 칼슘채널차단제(calcium channel blocker; flunarizine 등)를 사용하는 것이 타약제, 위약 또는 치료하지 않는 것에 비해 두통의 완화에 효과적인가?
	\item Episodic migraine환자에서 예방치료로 안지오텐신전환효소억제제(angiotensin converting enzyme inhibitor)나 안지오텐신수용체차단제(angiotensin receptor blocker; candesartan 등)를 사용하는 것이 타약제, 위약 또는 치료하지 않는 것에 비해 두통의 완화에 효과적인가?
	\item Episodic migraine환자에서 예방치료로 항우울제(anti-depressant; amitryptiline 등)를 사용하는 것이 타약제, 위약 또는 치료하지 않는 것에 비해 두통의 완화에 효과적인가?
	\item Episodic migraine환자에서 예방치료로 항경련제(anti-epileptic agent; divalproex sodium, sodium valproate, topiramate 등)를 사용하는 것이 타약제, 위약 또는 치료하지 않는 것에 비해 두통의 완화에 효과적인가?
	\item Episodic migraine환자에서 예방치료로 타약제, 위약 또는 치료하지 않는 것에 비해 두통의 완화에 효과적인 기타약제는 (항히스타민제 (eg. cyproheptadine), antithrombotics (eg. coumadin) 등) ?
	\item 월경편두통(Menstrual migraine)이나 월경관련편두통(Menstrually related migraine)환자에서 트립탄제제(triptan)의 예방적 사용이 타약제, 위약 또는 치료하지 않는 것에 비해 두통의 완화에 효과적인가?
\end{enumerate}
\color{black}

\begin{enumerate}
	\item 저출혈위험의 시술및 수술을 고려하고 있는 허혈성 뇌경색환자에서, 항혈전제를 어떻게 중지하는 것(중지 기간, 재시작 시점, bridging therapy여부 등)이 출혈및 혈전색전증의 위험을 최소화하는데 효과적인가?
	\item 고출혈위험의 시술및 수술을 고려하고 있는 허혈성 뇌경색환자에서, 항혈전제를 어떻게 중지하는 것(중지 기간, 재시작 시점, bridging therapy여부 등)이 출혈및 혈전색전증의 위험을 최소화하는데 효과적인가?
\end{enumerate}
 
 \color{red}
 Bleeding risk: \\
 high vs. low\\
 
 Thromboembolic risk: \\
 acute or recent stroke, stenting (bare metal stent, DES?) within 3 months ?, \\
 LAD with $>$50\% stenosis, \\
 Af, prothetic valve, \\
 CVT?
\end{document}