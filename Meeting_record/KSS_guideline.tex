% !TEX encoding = UTF-8 Unicode

%%%%%%%%%%%%%%%%%%%%%%%%%%%%%%%%%%%%%%%%%
% Daily Laboratory Book
% LaTeX Template
%
% This template has been downloaded from:
% http://www.latextemplates.com
%
% Original author:
% Frank Kuster (http://www.ctan.org/tex-archive/macros/latex/contrib/labbook/)
%
% Important note:
% This template requires the labbook.cls file to be in the same directory as the
% .tex file. The labbook.cls file provides the necessary structure to create the
% lab book.
%
% The \lipsum[#] commands throughout this template generate dummy text
% to fill the template out. These commands should all be removed when 
% writing lab book content.
%
% HOW TO USE THIS TEMPLATE 
% Each day in the lab consists of three main things:
%
% 1. LABDAY: The first thing to put is the \labday{} command with a date in 
% curly brackets, this will make a new page and put the date in big letters 
% at the top.
%
% 2. EXPERIMENT: Next you need to specify what experiment(s) you are 
% working on with an \experiment{} command with the experiment shorthand 
% in the curly brackets. The experiment shorthand is defined in the 
% 'DEFINITION OF EXPERIMENTS' section below, this means you can 
% say \experiment{pcr} and the actual text written to the PDF will be what 
% you set the 'pcr' experiment to be. If the experiment is a one off, you can 
% just write it in the bracket without creating a shorthand. Note: if you don't 
% want to have an experiment, just leave this out and it won't be printed.
%
% 3. CONTENT: Following the experiment is the content, i.e. what progress 
% you made on the experiment that day.
%
%%%%%%%%%%%%%%%%%%%%%%%%%%%%%%%%%%%%%%%%%

%----------------------------------------------------------------------------------------
%	PACKAGES AND OTHER DOCUMENT CONFIGURATIONS
%----------------------------------------------------------------------------------------

\documentclass[idxtotoc,hyperref,openany, oneside]{labbook} % 'openany' here removes the gap page between days, erase it to restore this gap; 'oneside' can also be added to remove the shift that odd pages have to the right for easier reading

\usepackage[ 
  backref=page,
  pdfpagelabels=true,
  plainpages=false,
  colorlinks=true,
  bookmarks=true,
  pdfview=FitB]{hyperref} % Required for the hyperlinks within the PDF
  
\usepackage{booktabs} % Required for the top and bottom rules in the table
\usepackage{float} % Required for specifying the exact location of a figure or table
\usepackage{graphicx} % Required for including images
\usepackage{lipsum} % Used for inserting dummy 'Lorem ipsum' text into the template
\usepackage{kotex}
\usepackage{tikz}
\usepackage{setspace}
\usepackage{hyperref}

\newcommand{\attend}[1]{\textbf{참석자:} \emph{#1}\\} 
\newcommand{\place}[1]{\textbf{장소†Œ:} \emph{#1}\\} 

\newcommand{\HRule}{\rule{\linewidth}{0.5mm}} % Command to make the lines in the title page
\setlength\parindent{0pt} % Removes all indentation from paragraphs

%----------------------------------------------------------------------------------------
%	DEFINITION OF EXPERIMENTS
%----------------------------------------------------------------------------------------

%\newexperiment{example}{This is an example experiment}
%\newexperiment{example2}{This is another example experiment}
%\newexperiment{example3}{This is yet another example experiment}
%\newexperiment{table}{This shows a sample table}
%\newexperiment{shorthand}{Description of the experiment}

%---------------------------------------------------------------------------------------

\begin{document}

%----------------------------------------------------------------------------------------
%	TITLE PAGE
%----------------------------------------------------------------------------------------

\frontmatter % Use Roman numerals for page numbers
\title{
\begin{center}
\HRule \\[0.4cm]
{\Huge \bfseries 두통진료지침 - 편두통 예방약물}\\[0.2cm] % Degree
\HRule \\[1.5cm]
\end{center}
}
\author{\LARGE 박 광 열 \\ \\ \LARGE kwangyeol.park@gmail.com \\[2cm]} 
\date{Beginning 21 November 2016} % Beginning date
\maketitle

\tableofcontents

\mainmatter % Use Arabic numerals for page numbers

\setstretch{1.2}

%----------------------------------------------------------------------------------------
\labday{Monday, 21 November 2016}
\attend{김병수, 김병건, 김원주, 정재면, 박광열, 박동아, 송태진, 서종근, 이미지, 정필욱}

\experiment{두통진료지침의 작성 방안•ˆ}
	\begin{itemize}
		\item Hydrid method
		\item Review of current guidelines 
		\item Extract research questions and reference articles
		\item Assess the quality of guidelines
	\end{itemize}

\experiment{다음 모임시까지 해야 할 일}
	\begin{itemize}
		\item Guideline을 위원들에게 배정하여 research question과 reference article 정리
	\end{itemize}
	
\experiment{Next meeting}
Time: Dec. 19, 2016 (Mon)\\
Place: not determined
	
%%----------------------------------------------------------------------------------------
%\labday{Monday, 19 December 2016}
%\attend{김병수, 김병건, 김원주, 정재면, 박광열, 박동아, 손종희, 송태진, 서종근, 이미지, 정필욱, 최윤주}
%\place{신경과학회 사무실}
%
%\experiment{기존 진료지침 검색 및 선택}
%	\begin{itemize}
%		\item Selection criteria
%			\begin{itemize}
%				\item Since 2012
%				\item Written in English
%				\item By Multi-disciplinary team
%				\item Evidence-based
%			\end{itemize}
%		\item 상기 조건외에 추가조건은 추후 결정: 박광열, 박동아.
%		\item 2인의 위원이 독자적으로 선택한 후, 상이한 부분은 consensus에 의해 결정.
%		\item 검색전략에 대한 기술 필요
%	\end{itemize}
%	
%\experiment{기존 진료지침 선택 후}
%	\begin{itemize}
%		\item 각각의 진료지침을 2인이 review
%		\item 권고안(핵심질문)과 그 근거가 되는 저널 선정: template이 필요
%		\item 전체 진료지침을 모은 후, 위원들이 소규모로 모여 핵심질문(PICO형식) 선정
%		\item 필요시 기존 진료지침의 발간후 추가된 증거에 대한 RCT, meta analysis등 추가 검색 필요
%	\end{itemize}
%
%\experiment{Protocol 정리}
%전체 protocol 정리 (박동아, first author)하여 출간.
%
%\experiment{Next meeting}
%소규모 미팅: 23 January 2017. 학회 사무실\\
%전체 미팅: 27 February 2017. 학회 사무실
%
%%----------------------------------------------------------------------------------------
%\labday{After meeting (19 December 2016)}
%
%\experiment{기존 진료지침 검색 및 선택}
%	\begin{itemize}
%		\item \href{RefFiles/20170112/searchingProcess20170112LeeMJ}{Search Process by LeeMJ}
%		\item 9개의 진료지침 (이미지, 송태진 선택)
%		\begin{itemize}
%			\item 2012 AAN AHS guideline update for migraine prophylaxis Neurology (김병수, 정필욱)
%			\item 2012 Canadian guideline for migraine prophylaxis Can J neurol sci (이미지, 손종희)
%			\item 2012 Croatia guideline (송태진, 손종희)
%			\item 2012 Danish Guidelines JHP (서종근, 최윤주)
%			\item 2012 French guidelines revised JHP (정필욱, 김병수ˆ˜)
%			\item 2012 Italian Guidelines revised version -JHP supple (최윤주, 서종근) 
%			\item ICSI guideline 2013 (original version, full-text) (송태진, 김병수)
%			\item NICE guideline for headache in over 12s (2012) (2015 update) (이미지, 김병건)
%			\item Practice guideline update summary fot Botulinum neurotoxin (AAN) 2016 neurology (이미지)
%		\end{itemize}
%	\end{itemize}
%	
%\experiment{지침 개발 프로토콜 - 박동아}
%	두통지침개발프로토콜 ver 1.0
%	
%%----------------------------------------------------------------------------------------
%\labday{Monday, 23 January 2017}
%\attend{김병건, 정재면, 박광열, 손종희, 송태진, 서종근, 이미지, 정필욱, 최윤주}
%\place{신경과학회 사무실}
%
%\experiment{Key question 발췌 및 토의}
%	\begin{itemize}
%		\item 9개의 진료지침 검토
%		\begin{itemize}
%			\item 2012 AAN AHS guideline update for migraine prophylaxis Neurology (김병수, 정필욱)
%			\item 2012 Canadian guideline for migraine prophylaxis Can J neurol sci (이미지, 손종희)
%			\item 2012 Croatia guideline (송태진, 손종희)
%			\item 2012 Danish Guidelines JHP (서종근, 최윤주)
%			\item 2012 French guidelines revised JHP (정필욱, 김병수ˆ˜)
%			\item 2012 Italian Guidelines revised version -JHP supple (최윤주, 서종근) 
%			\item ICSI guideline 2013 (original version, full-text) (송태진, 김병수)
%			\item NICE guideline for headache in over 12s (2012) (2015 update) (이미지, 김병건)
%			\item Practice guideline update summary fot Botulinum neurotoxin (AAN) 2016 neurology (이미지)
%		\end{itemize}
%		\item 논의된 key question을 정리(박광열)하여 전체 회람후 의견 수렴 예정.
%		\item 이후 선정된 key question을 이용하여 진료지침 검색 및 평가 진행 예정.
%	\end{itemize}
%	
%%----------------------------------------------------------------------------------------
%\labday{Monday, 13 February 2017}
%\attend{김병건, 김원주, 정재면, 박광열, 이유경}
%\place{서울파이낸스센터 메이징에이}
%
%\experiment{향후 진행과정에 대한 검토 및 논의}
%	\begin{itemize}
%		\item 2012년 AAN guideline과 2012년 Canadian guideline을 중심으로 guideline 평가 연습 실시
%		\item 국내에서 허가가 되지 않은 약에 대해서도 의견을 제시 할 수 있음(이유경).
%		\item NICE stroke guideline과 Canada guideline을 모범적인 사례로 연구해 볼 만 함 (이유경).
%		\item Guideline search는 NECA에 의뢰/고려대 의뢰/학교 도서관 사서에게 의뢰등을 고려.
%		\item 진료지침에는 환자의 가치평가, 전문가의 의견, 과학적 증거등이 잘 어우러져야 함.
%		\item 기존의 guideline들이 주로 2012년에 나왔으므로, 선택한 guideline의 검색식을 이용하여 추가적인 논문 search가 필요함.
%		\item 2012 AAN guideline의 web site를 방문하여 추가적인 material에 대한 조사가 필요함.
%	\end{itemize}
%	
%\experiment{Next meeting}
%09 March  2017. 학회 사무실\\
%위원들에게 AAN and Canadian guideline 배포하고 평가를 미리 시행해 오도록 공지(박광열).
%
%%----------------------------------------------------------------------------------------
%\labday{Tuesday, 09 March 2017}
%\attend{김병건, 김원주, 정재면, 박광열, 이유경, 정필욱, 송태진, 김병수, 이미지, 서종근 (무순)}
%\place{대한신경과학회 사무실}
%
%\experiment{AGREE II 를 이용한 AAN/Canadian guideline review}
%	\begin{itemize}
%		\item AGREE 2 평가도구 (ppt) 참고
%		\item 향후 위원으로 가정의학과 전문의, 1차 진료담당 의사 포함 고려
%		\item Patient's opinion/view를 향후 연구주제로 고려해 볼 만 함.
%		\item guideline monitoring을 위해 심평원자료 이용/신경과 전문의 survey 고려
%		\item 다음 모임에서 잘 만들어진 guideline review하기로 함
%	\end{itemize}
%	
%\experiment{Next meeting}
%11 April  2017. 학회 사무실\\
%Agenda\\
%\begin{itemize}
%	\item 특강: 잘 개발된 임상진료지침이란? (이유경)
%	\item 미리 배포된 guideline review (진료지침위원)
%	\item Clinical question에 대한 논의 (전원)
%\end{itemize}
%
%%----------------------------------------------------------------------------------------
%\labday{Tuesday, 11 April 2017}
%\attend{김병건, 김원주, 정재면, 박광열, 이유경, 정필욱, 송태진, 최윤주, 손종희(무순)}
%\place{대한신경과학회 사무실}
%
%\experiment{특강: 잘 개발된 임상진료지침이란 (이유경)} 
%
%\experiment{현재까지의 진행과 향후 계획}
%
%\begin{itemize}
%	\item 자료: https://github.com/sbaram1/Guideline\_Migraine\_Prevention
%	\item 향후 실무위원을 아래 세개의 팀으로 구성
%	\begin{description}
%		\item[A team] 김병수, 서종근
%		\item[B team] 손종희, 송태진
%		\item[C team] 이미지, 정필욱, 최윤주
%	\end{description}
%	\item 각 팀당 key questions을 2-3개씩 맡아서 검색/평가/draft작성등을 진행.
%	\item 검색, 근거표등은 고려대 의뢰를 고려.
%\end{itemize}
%
%\experiment{Next meeting}
%2017년 8월경

%%-----------------------------------------
%
%\experiment{example}
%
%\lipsum[6]
%
%%-----------------------------------------
%
%\experiment{example2}
%
%\lipsum[7]
%
%%----------------------------------------------------------------------------------------
%%	FORMULAE AND MEDIA RECIPES
%%----------------------------------------------------------------------------------------
%
%\labday{} % We don't want a date here so we make the labday blank
%
%\begin{center}
%\HRule \\[0.4cm]
%{\huge \textbf{Formulae and Media Recipes}}\\[0.4cm] % Heading
%\HRule \\[1.5cm]
%\end{center}
%
%%----------------------------------------------------------------------------------------
%%	MEDIA RECIPES
%%----------------------------------------------------------------------------------------
%
%\newpage
%
%\huge \textbf{Media} \\ \\
%
%\normalsize \textbf{Media 1}\\
%\begin{table}[H]
%\begin{tabular}{l l l}
%\toprule
%\textbf{Compound} & \textbf{1L} & \textbf{0.5L}\\
%\toprule
%Compound 1 & 10g & 5g\\
%Compound 2 & 20g & 10g\\
%\bottomrule
%\end{tabular}
%\caption{Ingredients in Media 1.}
%\label{tab:med1}
%\end{table}
%
%%-----------------------------------------
%
%%\textbf{Media 2}\\ \\
%
%%Description
%
%%----------------------------------------------------------------------------------------
%%	FORMULAE
%%----------------------------------------------------------------------------------------
%
%\newpage
%
%\huge \textbf{Formulae} \\ \\
%
%\normalsize \textbf{Formula 1 - Pythagorean theorem}\\ \\
%$a^2 + b^2 = c^2$\\ \\
%
%%-----------------------------------------
%
%%\textbf{Formula X - Description}\\ \\
%
%%Formula
%
%%----------------------------------------------------------------------------------------

\end{document}